\documentclass[runningheads]{llncs}
%
\usepackage{graphicx}
\usepackage{xtab}
\usepackage{stmaryrd}
\usepackage{amsmath, amssymb}
% If you use the hyperref package, please uncomment the following line
% to display URLs in blue roman font according to Springer's eBook style:
% \renewcommand\UrlFont{\color{blue}\rmfamily}

% OUR MACROS
%%%%%%%%%%%%%%%%%%%%%%%%%%%%%%%%%%%%%%%%%%%%%%
%							References
%%%%%%%%%%%%%%%%%%%%%%%%%%%%%%%%%%%%%%%%%%%%%%

\newcommand{\chapref}[1]{Chapter~\ref{#1}}
\newcommand{\appref}[1]{Appendix~\ref{#1}}
\newcommand{\sectref}[1]{Section~\ref{#1}}
\newcommand{\subsectref}[1]{Subsection~\ref{#1}}
\newcommand{\figref}[1]{Figure~\ref{#1}}
\newcommand{\tabref}[1]{Table~\ref{#1}}
\newcommand{\egref}[1]{Example~\ref{#1}}
\newcommand{\eqnref}[1]{(\ref{#1})}
\newcommand{\thmref}[1]{Theorem~\ref{#1}}
\newcommand{\propref}[1]{Proposition~\ref{#1}}
\newcommand{\lemref}[1]{Lemma~\ref{#1}}
\newcommand{\defref}[1]{Definition~\ref{#1}}

\newcommand{\chapchapref}[2]{Chapters~\ref{#1} and \ref{#2}}
\newcommand{\sectsectref}[2]{Sections~\ref{#1} and \ref{#2}}
\newcommand{\figfigref}[2]{Figures~\ref{#1} and \ref{#2}}
\newcommand{\tabtabref}[2]{Tables~\ref{#1} and \ref{#2}}

\newcommand{\sectsectsectref}[3]{Sections~\ref{#1}, \ref{#2} and \ref{#3}}
\newcommand{\figfigfigref}[3]{Figures~\ref{#1}, \ref{#2} and \ref{#3}}

%%%%%%%%%%%%%%%%%%%%%%%%%%%%%%%%%%%%%%%%%%%%%%
%							Code Listings
%%%%%%%%%%%%%%%%%%%%%%%%%%%%%%%%%%%%%%%%%%%%%%

%% Core language
%\lstdefinelanguage{SaltyLang}{
% keywords=[1]{all, any, mutex}
% keywordstyle[1]=\color{Cyan},
% keywords=[2]{controller, where, input, output, enum, env_init, env_trans, env_liveness, sys_init, sys_trans, sys_liveness, def},
% keywordstyle=[2]\color{blue!90!black},%\bfseries,
% keywords=[3]{False, True},
% keywordstyle=[3]\color{ForestGreen},%\bfseries,
% keywords=[4]{all, any, mutex, leads_to, Bool, if, then, else},
% keywordstyle=[4]\color{Cyan},
% otherkeywords={', !, <, ->, <->, \, /, /\\, \\/},
% keywordstyle=\color{Cyan},
% identifierstyle=\color{black},
% sensitive=true,
% comment=[l]{--},
% commentstyle=\color{Magenta}
%}
%
%% Additional options
%\lstset{
% language=SaltyLang,
% basicstyle=\small\ttfamily,
% tabsize = 2,
% numbers=none,
% mathescape=false,
% showstringspaces=false
%}

%%%%%%%%%%%%%%%%%%%%%%%%%%%%%%%%%%%%%%%%%%%%%%
%							Theorem Environments
%%%%%%%%%%%%%%%%%%%%%%%%%%%%%%%%%%%%%%%%%%%%%%

%\newtheorem{lemma}{Lemma}
%\theoremstyle{definition} % turn off pesky italics
%\newtheorem{definition}{Definition}
%\newtheorem{example}{Example}
%\newtheorem{theorem}{Theorem}
%\newtheorem{proposition}{Proposition}
%\newtheorem{proof}{Proof}


%%%%%%%%%%%%%%%%%%%%%%%%%%%%%%%%%%%%%%%%%%%%%%
%							Definitions
%%%%%%%%%%%%%%%%%%%%%%%%%%%%%%%%%%%%%%%%%%%%%%

% Standard sets of numbers
\def\Nset{\mathbb{N}}
\def\Rset{\mathbb{R}}
\def\Zset{\mathbb{Z}}

% Reactive synthesis sets
\def\inputs{\mathcal{I}}
\def\outputs{\mathcal{O}}


% abbreviations
%\def\act{{Act}}
%\def\sinit{{\overline{s}}}

% logical operators
\def\limplies{\rightarrow}
\newcommand{\lequiv}{\leftrightarrow}
\newcommand{\ltrue}{\textrm{true}}
\newcommand{\lfalse}{\textrm{false}}
\newcommand{\lxor}{\oplus}

% temporal operators
\def\always{\square}
\def\until{\mathbin{\mathsf{U}}}
\def\wuntil{\mathbin{\mathsf{W}}}
\def\release{\mathbin{\mathsf{R}}}
\def\eventually{\lozenge}
\def\always{\square}
\def\lnext{\varbigcirc}

\def\palways{\square^{-1}}
\def\peventually{\lozenge^{-1}}
\def\past{\varbigcirc^{-1}}

\newcommand{\talways}[2]{\square^{#1..#2}}
\newcommand{\teventually}[2]{\lozenge^{#1..#2}}
\newcommand{\tuntil}[2]{\mathbin{\mathsf{U}}^{#1..#2}}
\newcommand{\trelease}[2]{\mathbin{\mathsf{R}}^{#1..#2}}


% model checking

%\def\true{{\mathit{true}}}
%\def\false{{\mathit{false}}}
%\def\AP{{\mathit{AP}}}
%\def\Dist{{\mathit{Dist}}}
%\def\Sat{{\mathit{Sat}}}
%\def\Steps{{\mathit{Steps}}}
%\def\sinit{{s_\mathit{init}}}
%\def\Prob{{\mathit{Prob}}}
%\def\Pr{{\mathit{Pr}}}
%\def\Path{{\mathit{Path}}}
%\def\Pathfin{{\mathit{FPath}}}
%\def\Pathful{{\mathit{Path}_\mathit{ful}}}
%\def\Adv{{\mathit{Adv}}}
%\def\calAfair{{{\cal A}_\mathit{fair}}}
%\def\sat{{\,\models\,}}
%\def\notsat{{\,\not\models\,}}
%\def\satAdv{{\,\models_\mathit{Adv}\,}}
%\def\notsatAdv{{\,\not\models_\mathit{Adv}\,}}
%\def\satfair{{\,\models_\mathit{fair}\,}}
%\def\next{{X\,}}
%\def\until{{\ {\cal U}\ }}
%\def\buntil{{\ {\cal U}^{\leq k}\ }}
%\def\tuntil{{\ {\cal U}^{\leq t}\ }}
%\def\sqlt{{\sqsubset}}
%\def\sqleq{{\sqsubseteq}}
%\def\sqgt{{\sqsupset}}
%\def\sqgeq{{\sqsupseteq}}
%\def\psmin{{p_s^\mathit{min}}}
%\def\psmax{{p_s^\mathit{max}}}
%\def\ptmin{{p_t^\mathit{min}}}
%\def\ptmax{{p_t^\mathit{max}}}
%\def\pspmin{{p_{s'}^\mathit{min}}}
%\def\pspmax{{p_{s'}^\mathit{max}}}
%\def\Syes{{S^{\mathit{yes}}}}
%\def\Sno{{S^{\mathit{no}}}}
%\def\Sqm{{S^?}}
%\def\done{{\mathit{done}}}
%\def\calPbp{{\cal P}_{\bowtie p}}
%\def\calSbp{{\cal S}_{\bowtie p}}
%\def\Var{{\mathit{Var}}}
%\def\Act{{\mathit{Act}}}
%\def\glob{{\mathit{glob}}}
%\def\ind{{\mathit{ind}}}
%\def\enc{{\mathit{enc}}}
%\def\etmcc{{ $\mathsf{E} {\;} \rotatebox{90}{$\mathsf{T}$} {\;} \mathsf{MC}^\mathsf{2}$ }}
%\def\implies{{\rightarrow}}

%%%%%%%%%%%%%%%%%%%%%%%%%%%%%%%%%%%%%%%%%%%%%%
%							Commands
%%%%%%%%%%%%%%%%%%%%%%%%%%%%%%%%%%%%%%%%%%%%%%


\newcommand{\pdist}[1]{\mathcal{D}({#1})}
\newcommand{\tfunction}{\longrightarrow}
\newcommand{\trans}[3]{#1 \overset{#2}{\longrightarrow} #3}
\newcommand{\allstates}{\overline{S}}

\newcommand{\allstratstwo}{\Sigma}
\newcommand{\strattwo}{\sigma}
\newcommand{\Rsetinf}{\Rset_{\pm\infty}}
\def\Rset{\mathbb{R}}
\def\Qset{\mathbb{Q}}
\newcommand{\dwc}{\textsf{dwc}}
\newcommand{\mydef}{\stackrel{\mbox{\rm {\tiny def}}}{=}}
\newcommand{\PONE}[1]{\textsf{Player~1}}
\newcommand{\PTWO}[1]{\textsf{Player~2}}
\newcommand{\PLAYER}[1]{\textsf{Player}~${#1}$}
\newcommand{\Exp}{\mathbb{E}}
%\renewcommand{\Pr}{\mathrm{Pr}}
\newcommand{\rew}{\mathsf{rew}}


\def\calPle{{\cal P}_{\le \lambda}}
\def\Pr{{\mathit{Pr}}}
\def\Prps{{\mathit{Pr}_{M^c}^{\sinit}}}


\newcommand{\salty}{Salty\xspace}
\newcommand{\enum}{{\texttt{enum}}\xspace}
\newcommand{\true}{{\texttt{True}}\xspace}
\newcommand{\false}{{\texttt{False}}\xspace}
\newcommand*\necessary{\mathord{\Box}}
\newcommand*\possible{\mathord{\Diamond}}
\newcommand{\sysinit}{\lstinline{sys_init}\xspace}%{{\texttt{sys\_init}}\xspace}
\newcommand{\envinit}{\lstinline{env_init}\xspace}%{{\texttt{env\_init}}\xspace}
\newcommand{\systrans}{\lstinline{sys_trans}\xspace}%{{\texttt{sys\_trans}}\xspace}
\newcommand{\envtrans}{\lstinline{env_trans}\xspace}%{{\texttt{env\_trans}}\xspace}
\newcommand{\sysliveness}{\lstinline{sys_liveness}\xspace}%{{\texttt{sys\_liveness}}\xspace}
\newcommand{\envliveness}{\lstinline{env_liveness}\xspace}%{{\texttt{env\_liveness}}\xspace}


\begin{document}
%%%%%%%%%%%%%%%%%%%%%%%%%%%%%%%%%%%%%%%%%%%%%%%%%%%%%%%%%%%%%%%%%%%%%%%%%%%%%%%%
% TITLE
%%%%%%%%%%%%%%%%%%%%%%%%%%%%%%%%%%%%%%%%%%%%%%%%%%%%%%%%%%%%%%%%%%%%%%%%%%%%%%%%
%
\title{End-to-End Verification of Initial and Transition Properties of GR(1) Designs in SPARK\thanks{Supported by AFRL contract FA8650-16-C-2642 and AFOSR grant RQCOR20-35.}}
%
\titlerunning{End-to-End Verification of GR(1) Designs in SPARK}
% If the paper title is too long for the running head, you can set
% an abbreviated paper title here
%
\author{First Author\inst{1}\orcidID{0000-1111-2222-3333} \and
Second Author\inst{2,3}\orcidID{1111-2222-3333-4444} \and
Third Author\inst{3}\orcidID{2222--3333-4444-5555}}
%
\authorrunning{F. Author et al.}
% First names are abbreviated in the running head.
% If there are more than two authors, 'et al.' is used.
%
\institute{Princeton University, Princeton NJ 08544, USA \and
Springer Heidelberg, Tiergartenstr. 17, 69121 Heidelberg, Germany
\email{lncs@springer.com}\\
\url{http://www.springer.com/gp/computer-science/lncs} \and
ABC Institute, Rupert-Karls-University Heidelberg, Heidelberg, Germany\\
\email{\{abc,lncs\}@uni-heidelberg.de}}
%
\maketitle              % typeset the header of the contribution
%
%%%%%%%%%%%%%%%%%%%%%%%%%%%%%%%%%%%%%%%%%%%%%%%%%%%%%%%%%%%%%%%%%%%%%%%%%%%%%%%%
% ABSTRACT
%%%%%%%%%%%%%%%%%%%%%%%%%%%%%%%%%%%%%%%%%%%%%%%%%%%%%%%%%%%%%%%%%%%%%%%%%%%%%%%%
\begin{abstract}
%The abstract should briefly summarize the contents of the paper in 150--250 words.
Manually designing control logic for reactive systems is time-consuming and error-prone. 
An alternative is to automatically generate controllers using ``correct-by-construction'' synthesis approaches. 
Recently, there has been interest in synthesizing controllers from Generalized Reactivity(1) or GR(1) specifications, 
since computational complexity is relatively low, and several tools now exist. 
However, though these tools implement synthesis approaches that are theoretically ``correct-by-construction,'' 
errors in tool implementation can lead to errors in synthesized controllers. 
We are therefore interested in ``end-to-end'' verification of synthesized controllers with respect to their original GR(1) specifications.
Toward this end we have modified Salty, a tool that produces executable software implementations of controllers 
from GR(1) specifications in a variety of programming languages, to produce implementations in SPARK.
SPARK is both a language and associated set of verification tools, and so its use enables the desired ``end-to-end'' verification of a 
subset of properties comprising GR(1) specifications, in this case system initial and transition properties. 
In this paper, we discuss how we encode synthesized controllers and the necessary contracts and assertions needed for 
SPARK to verify the aforementioned properties automatically, i.e. without requiring additional annotations from the user. 
Since GR(1) specifications encode assumptions about the environment in which the controlled system operates, 
we also discuss how we handle these assumptions in the implementation, since there are several reasonable choices for how they could be handled. 
Finally, we discuss possible approaches for verifying the full set of GR(1) properties, i.e. including liveness.

\keywords{Reactive synthesis \and end-to-end verification \and functional verification.}
\end{abstract}

%%%%%%%%%%%%%%%%%%%%%%%%%%%%%%%%%%%%%%%%%%%%%%%%%%%%%%%%%%%%%%%%%%%%%%%%%%%%%%%%
% INTRODUCTION
%%%%%%%%%%%%%%%%%%%%%%%%%%%%%%%%%%%%%%%%%%%%%%%%%%%%%%%%%%%%%%%%%%%%%%%%%%%%%%%%
\section{Introduction}

Reactive systems must be capable of correctly responding to various inputs, 
e.g. originating from human users or events in the system's operational environment. 
The process of manually designing the control logic for such systems is both time-consuming and error prone. 
An alternative is to use ``correct-by-construction'' synthesis approaches to automatically generate 
the design of a system's control logic directly from specifications, which can 
reduce both design time and the likelihood of errors 
\cite{alur2016compositional}, \cite{fainekos2009temporal}, \cite{guo2014cooperative}, \cite{kupermann2001synthesizing}. 
In general, \emph{synthesis} is the process of automatically generating a design from a specification. 
More specifically, \emph{reactive synthesis} approaches generate designs in the context of an uncontrolled environment, 
assumptions about which are encoded in the specification.
Reactive synthesis approaches tend to have high computational complexity,  
so there is particular interest in synthesis from Generalized Reactivity(1) or GR(1) specifications, 
the complexity of which is only polynomial in the size of the game graph encoded by the specification \cite{bloem2012}.
Synthesis from GR(1) specifications has been used to generate digital circuits \cite{ehlers2012symbolically} 
and controllers for aircraft power distribution \cite{xu2012case}, 
software-defined networks \cite{wang2013automated}, ground robots \cite{kress2007s},
and teams of unmanned vehicles \cite{apker2016}, to name a few. 

A GR(1) specification $\varphi$ takes the form $\varphi = \varphi^e \limplies \varphi^s$, 
where $\varphi^e$ encodes assumptions about the \emph{environment} in which a system is to operate,
and $\varphi^s$ encodes guarantees the \emph{system} should make under those assumptions \cite{Ehlers2016}. 
More specifically, $\varphi$ takes the form 
$\varphi = (\varphi^e_i \land \varphi^e_t \land \varphi^e_l)  \limplies (\varphi^s_i \land \varphi^s_t \land \varphi^s_l)$,
where $\varphi^e_i$ and $\varphi^s_i$ are \emph{initial} properties, 
$\varphi^e_t$ and $\varphi^s_t$ are \emph{transition} or safety properties, and 
$\varphi^e_l$ and $\varphi^s_l$ are \emph{liveness} properties. 
For inputs in the set $\inputs$ controlled by the environment 
and outputs in the set $\outputs$ produced by the system, terms are:

\vspace{0.5em}

\noindent \begin{xtabular}{p{.045\columnwidth}p{.015\columnwidth}p{.860\columnwidth}}
 $\varphi^e_i$, $\varphi^s_i$ & - & Boolean formulas over $\inputs$ and $\outputs$, respectively, that characterize the initial state of the environment and system.
\end{xtabular}

\noindent \begin{xtabular}{p{.045\columnwidth}p{.015\columnwidth}p{.860\columnwidth}}
 $\varphi^e_t$, $\varphi^s_t$ & - &  Formulas of the form $\bigwedge_{j \in J} \always B_j$, where each $B_j$ is a Boolean combination of variables from $\inputs \cup \outputs$ and expressions of the form $\lnext v$, where $v \in \inputs$ for $\varphi^e_t$ and $v \in \inputs \cup \outputs$ for $\varphi^s_t$. These encode properties that should always hold as well as rules for how inputs and outputs are allowed to change based on current input and output values.
\end{xtabular}

\noindent \begin{xtabular}{p{.045\columnwidth}p{.015\columnwidth}p{.860\columnwidth}}
$\varphi^e_l$, $\varphi^s_l$ & - & Formulas of the form $\bigwedge_{j \in J} \always \eventually B_j$, where each $B_j$ is a Boolean formula over $\inputs \cup \outputs$. These encode properties that should hold infinitely often.
\end{xtabular}

\vspace{0.5em}

\noindent It is assumed that at each time step, the environment chooses an input from $\inputs$, and then the system chooses an output from $\outputs$ in response.

Several tools for synthesis from GR(1) specifications are available. 
For example, RATSY \cite{bloem2010ratsy} has a focus on circuit design and can synthesize designs encoded in 
BLIF (Berkeley Logic Interchange Format), Verilog, and HIF (HDL Intermediate Format). 
Similarly, Anzu \cite{jobstmann2007anzu} produces circuit designs in Verilog.
LTLMoP \cite{finucane2010ltlmop} is focused on control of robots modeled as hybrid systems, 
and it synthesizes designs as hybrid controllers along with handler modules to help connect controllers to simulated or real-world systems.
Slugs \cite{Ehlers2016} is architected to allow users to easily tailor the synthesis algorithm for specific applications,
 e.g. to optimize criteria such as quick response, cost-optimality, and error-resilience, and produces mathematical representations of controller designs.
Salty \cite{elliott2019salty} provides a front-end to Slugs that makes specifications easier to write and debug and 
a back-end that turns controller designs into executable sofware implementations in a variety of programming languages.

Though all of these tools implement synthesis algorithms that are theoretically ``correct-by-construction,'' 
tool implementation errors could result in errors in synthesized controllers. 
We are therefore interested in ``end-to-end'' verification of synthesized controllers with respect to their original GR(1) specifications. 
Toward this end, we have extended the Salty tool to produce software implementations of synthesized controller designs in SPARK. 
SPARK is both a programming language with a specification language and associated verification toolset \cite{hoang2015spark}. 
Though SPARK aims to perform fully automated verification, often the user must help guide the underlying provers through 
annotations such as assertions and loop invariants in the code. 
However, since controller designs synthesized from GR(1) specifications follow a regular structure, it was our hope that we could 
synthesize both the controller logic and the annotations necessary to automatically prove that controllers meet their functional specifications in SPARK. 
To date, we have largely achieved this goal for a subset of properties comprising GR(1) specifiations, 
i.e. system initial and transition properties, for moderately-sized controllers.
In what follows, in \sectref{sec:implementation} we discuss the SPARK implementation. 
In \sectref{sec:caseStudies}, we demonstrate the efficacy of our approach on examples pulled from a variety of sources.
In \sectref{sec:discussion}, we discuss possible approaches for proving liveness properties as well as design choice alternatives, 
e.g. for dealing with environment specifications.
We end with concluding remarks in 

%%%%%%%%%%%%%%%%%%%%%%%%%%%%%%%%%%%%%%%%%%%%%%%%%%%%%%%%%%%%%%%%%%%%%%%%%%%%%%%%
% IMPLEMENTATION and Verification in SPARK
%%%%%%%%%%%%%%%%%%%%%%%%%%%%%%%%%%%%%%%%%%%%%%%%%%%%%%%%%%%%%%%%%%%%%%%%%%%%%%%%
\section{Implementation and Verification in SPARK}
\label{sec:implementation}

%%%%%%%%%%%%%%%%%%%%%%%%%%%%%%%%%%%%%%%%%%%%%%%%%%%%%%%%%%%%%%%%%%%%%%%%%%%%%%%%
% CASE STUDIES
%%%%%%%%%%%%%%%%%%%%%%%%%%%%%%%%%%%%%%%%%%%%%%%%%%%%%%%%%%%%%%%%%%%%%%%%%%%%%%%%
\section{Case Studies}
\label{sec:caseStudies}



%%%%%%%%%%%%%%%%%%%%%%%%%%%%%%%%%%%%%%%%%%%%%%%%%%%%%%%%%%%%%%%%%%%%%%%%%%%%%%%%
% CASE STUDIES
%%%%%%%%%%%%%%%%%%%%%%%%%%%%%%%%%%%%%%%%%%%%%%%%%%%%%%%%%%%%%%%%%%%%%%%%%%%%%%%%
\section{Discussion}
\label{sec:discussion}

%%%%%%%%%%%%%%%%%%%%%%%%%%%%%%%%%%%%%%%%%%%%%%%%%%%%%%%%%%%%%%%%%%%%%%%%%%%%%%%%
% CONCLUSIONS
%%%%%%%%%%%%%%%%%%%%%%%%%%%%%%%%%%%%%%%%%%%%%%%%%%%%%%%%%%%%%%%%%%%%%%%%%%%%%%%%
\section{Conclusions}
\label{sec:conclusions}

%
% ---- Bibliography ----
%
% BibTeX users should specify bibliography style 'splncs04'.
% References will then be sorted and formatted in the correct style.
%
\bibliographystyle{splncs04}
\bibliography{bibfile}

\end{document}
